% a mashup of hipstercv, friggeri and twenty cv
% https://www.latextemplates.com/template/twenty-seconds-resumecv
% https://www.latextemplates.com/template/friggeri-resume-cv

\documentclass[lighthipster]{simplehipstercv}
% available options are: darkhipster, lighthipster, pastel, allblack, grey, verylight, withoutsidebar
% withoutsidebar
\usepackage[utf8]{inputenc}
\usepackage[default]{raleway}
\usepackage[margin=1cm, a4paper]{geometry}


%------------------------------------------------------------------ Variablen

\newlength{\rightcolwidth}
\newlength{\leftcolwidth}
\setlength{\leftcolwidth}{0.23\textwidth}
\setlength{\rightcolwidth}{0.75\textwidth}

%------------------------------------------------------------------
\title{VALENTIN LEVCHENKO}
\author{\LaTeX{} Ninja}
\date{Octobre 2025}

\pagestyle{empty}
\begin{document}


\thispagestyle{empty}
%-------------------------------------------------------------

\section*{Start}

\simpleheader{headercolour}{Valentin}{Levchenko}{MSc en Mathématiques}{white}



%------------------------------------------------

% this has to be here so the paracols starts..
\subsection*{}
\vspace{4em}

\setlength{\columnsep}{1.5cm}
\columnratio{0.23}[0.75]
\begin{paracol}{2}
\hbadness5000
%\backgroundcolor{c[1]}[rgb]{1,1,0.8} % cream yellow for column-1 %\backgroundcolor{g}[rgb]{0.8,1,1} % \backgroundcolor{l}[rgb]{0,0,0.7} % dark blue for left margin

\paracolbackgroundoptions

% 0.9,0.9,0.9 -- 0.8,0.8,0.8


\footnotesize
{\setasidefontcolour
\flushright
\begin{center}
    \roundpic{Levchenko_Valentin_Photo.jpg}
\end{center}


\faLinkedin~vallevchenko \hspace{1.5em}\\
\faGithub~ValentinLevchenko \hspace{1.5em}\\
\faInstagram~valentin.lv4k\\
\bigskip


\bg{cvgreen}{white}{À propos de moi}\\[0.5em]

{\footnotesize
Étudiant en Master de Mathématiques à l’Université de Strasbourg, spécialisé en mathématiques appliquées et modélisation numérique, 
je suis passionné par le calcul scientifique et les mathématiques tropicales et ouvert à des opportunités en recherche ou en 
data science. J'apprécie également l'enseignement et le mentorat.\\}
\bigskip

\bg{cvgreen}{white}{Informations personnelles} \\[0.5em]
Valentin Levchenko\\
Nationalité : russe\\
Date de naissance: 16/10/1995



\bigskip

\bg{cvgreen}{white}{Domaines de spécialisation} \\[0.5em]

Calcul scientifique
Modélisation numérique
Analyse de données
Enseignement et mentorat académique

\bigskip

\bg{cvgreen}{white}{Centres d’intérêt}\\[0.5em]

Mathématiques tropicales\\
Python\\
Tests logiciels\\
%\texttt{Mathématiques tropicales} 
%~/~ \texttt{Programmation scientifique (Python)}
%~/~ \texttt{Tests logiciels}
\bigskip

\phantom{turn the page}

\phantom{turn the page}
}
%-----------------------------------------------------------
\switchcolumn

\small
\section*{Expérience en recherche scientifique}

\begin{tabular}{r| p{0.5\textwidth} c}
    \cvevent{2022}{Mathématiques tropicales et Calcul Scientifique}
    {Stage}{SCORE (Conseil Scientifique pour la Recherche et l’Ingénierie), Strasbourg, France}{
        \begin{itemize}
            \item Étudié les problèmes d’inversion d’ensembles avec des intervalles tropicaux.
            \item Appliqué des méthodes de modélisation à des problèmes de mathématiques industrielles.
        \end{itemize}}{NoLogo.png}\\
    \cvevent{2022}{Asymptotiques à court terme pour le noyau de la chaleur sur des graphes finis et des variétés riemanniennes}
    {Mémoire de Master}{Université Georg August, Göttingen, Allemagne}{
        \begin{itemize}
            \item Analysé les méthodes de calcul des distances géodésiques sur des domaines plats et courbés.
            \item Compilé et synthétisé les résultats existants concernant les bornes supérieures et inférieures du noyau de chaleur sous différentes conditions.
        \end{itemize}}{goettingen.jpg}\\
    \cvevent{2018}{Diagramme de phase des gaz ultra-froids polarisés piégés}
    {Mémoire de Licence}{NRU-HSE, Moscou, Russie}{
        \begin{itemize}
            \item Modélisé des réseaux optiques et effectué des calculs à l’aide de la bibliothèque Python ALPS (Applications et Bibliothèques pour les Simulations en Physique).
            \item Implémenté un algorithme de calcul des diagrammes de phase.
        \end{itemize}}{HSE.jpeg}\\
    \cvevent{2017--2018}{Groupe de Recherche Scientifique "Modélisation des Phénomènes Collectifs dans les Systèmes à Particules Multiples"}
    {Assistant de laboratoire}{NRU-HSE, Moscou, Russie}{
        \begin{itemize}
            \item Simulé les tests de déformation et d’endurance des matériaux.
            \item Réalisé des simulations des expériences de déformation des matériaux (ex. : traction, flexion) à l’aide de la méthode des éléments finis.
        \end{itemize}}{HSE.jpeg}
\end{tabular}
\vspace{3em}

\begin{minipage}[t]{0.47\textwidth}
\section*{Formation universitaire}
\begin{tabular}{r p{0.72\textwidth} c}
    \cvdegree{2025}{Master}{Calcul Scientifique et Mathématique de l’Innovation}{Université de Strasbourg, France \color{headerblue}}{}{unistra.png} \\
    \cvdegree{2023}{Master}{Mathématiques}{Université Georg August, Göttingen, Allemagne \color{headerblue}}{}{goettingen.jpg} \\
    \cvdegree{2018}{Licence}{Mathématiques Appliquées}{Université Nationale de la Recherche “École supérieure d'économie”, Moscou, Russie \color{headerblue}}{}{HSE.jpeg}
\end{tabular}
\end{minipage}\hfill
\begin{minipage}[t]{0.22\textwidth}
\section*{Programmation}
\begin{tabular}{r @{\hspace{0.em}}l}
     \bg{skilllabelcolour}{iconcolour}{Python} &  \barrule{0.55}{0.5em}{cvgreen}\\
     \bg{skilllabelcolour}{iconcolour}{C/C++} & \barrule{0.45}{0.5em}{cvgreen} \\
     \bg{skilllabelcolour}{iconcolour}{Java} & \barrule{0.25}{0.5em}{cvpurple} \\
     \bg{skilllabelcolour}{iconcolour}{Scilab} & \barrule{0.25}{0.5em}{cvpurple} \\
     \bg{skilllabelcolour}{iconcolour}{Matlab} & \barrule{0.35}{0.5em}{cvpurple} \\
     \bg{skilllabelcolour}{iconcolour}{Octave} & \barrule{0.3}{0.5em}{cvpurple} \\
     \bg{skilllabelcolour}{iconcolour}{\LaTeX} & \barrule{0.35}{0.5em}{cvpurple} \\
\end{tabular}
\end{minipage}

\section*{Expérience Pedagogique}
\begin{tabular}{r| p{0.5\textwidth} c}
    \cvevent{2017--2018}{Algèbre et Géométrie Analytique - Licence 1}{Assistant pédagogique en enseignement supérieur}{NRU-HSE, Moscou, Russie}{
        \begin{itemize}
            \item Modélisé des réseaux optiques et effectué des calculs à l’aide de la bibliothèque Python ALPS (Applications et Bibliothèques pour les Simulations en Physique).
            \item Implémenté un algorithme de calcul des diagrammes de phase.
        \end{itemize}
    }{HSE.jpeg} 
\end{tabular}
\vspace{3em}

\begin{minipage}[t]{0.3\textwidth}

\section*{Langues}
\begin{tabular}{l | ll}
    \textbf{Anglais} & C2 & \pictofraction{\faCircle}{cvgreen}{4}{black!30}{0}{\tiny} \\
    \textbf{Français} & B2 & \pictofraction{\faCircle}{cvgreen}{3}{black!30}{1}{\tiny} \\
    \textbf{Allemand} & C1 & \pictofraction{\faCircle}{cvgreen}{3}{black!30}{1}{\tiny} \\
    \textbf{Espagnol} & B1 & \pictofraction{\faCircle}{cvgreen}{1}{black!30}{3}{\tiny} \\
    \textbf{Russe} & C2 & {\phantom{x}\footnotesize langue maternelle} 
\end{tabular}
\bigskip

\end{minipage}\hfill
\begin{minipage}[t]{0.3\textwidth}
\section*{Certifications}
\begin{tabular}{>{\footnotesize\bfseries}r >{\footnotesize}p{0.7\textwidth}}
    2018 & Cértificat d'anglais avancé, mention A \\
    2025 & Diplôme universitaire d’études françaises (DUEF) – niveau B2
\end{tabular}
\bigskip

\end{minipage}



\vfill{} % Whitespace before final footer

%----------------------------------------------------------------------------------------
%	FINAL FOOTER
%----------------------------------------------------------------------------------------
\setlength{\parindent}{0pt}
\begin{minipage}[t]{\rightcolwidth}
\begin{center}\fontfamily{\sfdefault}\selectfont \color{black!70}
{\small Valentin Levchenko \icon{\faEnvelopeO}{cvgreen}{} valentin.levchenko@etu.unistra.fr \icon{\faMapMarker}{cvgreen}{} Strasbourg \icon{\faPhone}{cvgreen}{} 07 69 42 23 65 {}
}
\end{center}
\end{minipage}

\end{paracol}

\end{document}
